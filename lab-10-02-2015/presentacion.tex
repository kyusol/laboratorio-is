\documentclass{beamer}
\usepackage[utf8]{inputenc}
\usepackage[spanish]{babel}
\usetheme{Goettingen}
\usecolortheme{default}
\useoutertheme{shadow}
\useinnertheme{rectangles}
\graphicspath{ {./figures/} }
\title[UNAM]{Ingeniería de Software}
\subtitle{Sesiones}
\author[Miguel]{Miguel Angel Piña Avelino}
\institute[UNAM]{
  Facultad de ciencias, UNAM
}
\date{\today}

\begin{document}

\frame{\titlepage}

\begin{frame}
  \frametitle{Índice}
  \tableofcontents
\end{frame}

\section{Introducción}
\begin{frame}
  \frametitle{Introducción}
  En esta sesión de laboratorio un poco más a detalles los siguientes conceptos:
  \begin{itemize}
    \item GIT
    \item Repositorios
  \end{itemize}
\end{frame}
\section{GIT}

\begin{frame}
  \frametitle{¿Qué es GIT?}
  GIT es un software de control de versiones diseñado por Linus Torvalds, pensando en la eficiencia y la confiabilidad del mantenimiento de versiones de aplicaciones cuando estas tienen un gran número de archivos de código fuente.
\end{frame}

\begin{frame}
  \frametitle{¿Qué es un sistema de control de versiones?}
  Se llama control de versiones a la gestión de los diversos cambios que se realizan sobre los elementos de algún producto o una configuración del mismo. Una versión, revisión o edición de un producto, es el estado en el que se encuentra el mismo en un momento dado de su desarrollo o modificación.
\end{frame}

\begin{frame}
  \frametitle{Comandos más comunes}
  \begin{itemize}
    \item git init
    \item git add
    \item git commit
    \item git push
    \item git fetch
    \item git pull
    \item git checkout rama
    \item git checkout -b nuevarama
  \end{itemize}
\end{frame}

\begin{frame}[fragile]
  \frametitle{Git for windows}
  Tutorial para windows (En inglés)
  \begin{verbatim}
    http://nathanj.github.io/gitguide/tour.html
  \end{verbatim}
\end{frame}

\section{Repositorios de código}
\begin{frame}
  \frametitle{¿Qué es un repositorio de código?}
  El repositorio es el lugar en el que se almacenan los datos actualizados e históricos de cambios, a menudo en un servidor. A veces se le denomina depósito o depot. Puede ser un sistema de archivos en un disco duro, un banco de datos, etc.
\end{frame}

\begin{frame}[fragile]
  \frametitle{Repositorios gratuitos}
  \begin{verbatim}
    www.bitbucket.org
    www.github.com
  \end{verbatim}
\end{frame}

\end{document}
%%% Local Variables:
%%% mode: latex
%%% TeX-master: t
%%% End:
